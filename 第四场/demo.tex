\documentclass{whutmod}
\usepackage[linesnumbered,ruled,lined]{algorithm2e}
\bibliographystyle{unsrt}
\team{10}
\membera{刘子川}
\joba{编程}
\memberb{程宇}
\jobb{建模}
\memberc{祁成}
\jobc{写作}
\hypersetup{
	colorlinks=true,
	linkcolor=black,citecolor=black
}


\newcommand{\upcite}[1]{\textsuperscript{\cite{#1}}}
%%%%%%%%%%%%%%%%%%%%%%%%%%%%%%%%%题目%%%%%%%%%%%%%%%%%%%%%%%%%%%%%%%%%%%%
\title{基于xxxxxxxx模型}
\tihao{1} 
\everymath{\displaystyle} 

\begin{document}

	\maketitle
	\thispagestyle{empty}
%%%%%%%%%%%%%%%%%%%%%%%%%%%%%%%%%摘要%%%%%%%%%%%%%%%%%%%%%%%%%%%%%%%%%%%%
	\begin{abstract}
		控制高压油管的压力变化对减小燃油量偏差,提高发动机工作效率具有重要意义。本文建立了基于质量守恒定理的微分方程稳压模型,采用二分法、试探法以及自适应权重的蝙蝠算法对模型进行求解。
		\vspace{6pt}	%空格
	
		针对问题一,建立基于质量守恒定律的燃油流动模型,考察单向阀开启时间对压力稳定性的影响。综合考虑压力与弹性模量、密度之间的关系,提出燃油压力-密度微分方程模型和燃油流动方程。本文采用改进的欧拉方法对燃油压力-密度微分方程求得数值解;利用二分法求解压力分布。综合考虑平均绝对偏差等反映压力稳定程度的统计量,求得直接稳定于100MPa的开启时长为\textbf{0.2955ms} ,在2s、5s内到达并稳定于150MPa时开启时长为\textbf{0.7795ms}、\textbf{0.6734ms},10s到达并稳定于150MPa的开启时长存在多解。最后对求解结果进行灵敏度分析、误差分析。
		\vspace{6pt}	%空格
	
		针对问题二,建立基于质量守恒定律的泵-管-嘴系统动态稳压模型,将燃油进入和喷出的过程动态化处理。考虑柱塞和针阀升程的动态变动,建立喷油嘴流量方程和质量守恒方程。为提高角速度求解精度,以凸轮转动角度为固定步长,转动时间变动步长,采用试探法粗略搜索与二分法精细搜索的方法求解,求得凸轮最优转动角速度\textbf{0.0283rad/ms(转速270.382转/分钟)},并得到该角速度下高压油管的密度、压力周期性变化图。对求解结果进行误差分析与灵敏度分析,考察柱塞腔残余容积变动对高压油管压力稳态的影响。
		\vspace{6pt}	%空格
	
		针对问题三,对于增加一个喷油嘴的情况,改变质量守恒方程并沿用问题二的模型调整供、喷油策略,得到最优凸轮转动角速度为\textbf{0.0522rad/ms(498.726转/分钟)};对于既增加喷油嘴又增加减压阀的情况,建立基于自适应权重的蝙蝠算法的多变量优化模型,以凸轮转动角速度、减压阀开启时长和关闭时长为参数,平均绝对偏差MAD为目标,在泵-管-嘴系统动态稳压模型的基础上进行求解,得到最优参数:\textbf{角速度0.0648 rad/ms(619.109转/分钟)}、减压阀的开启时长\textbf{2.4ms}和减压阀的关闭时长\textbf{97.6ms}。
		\vspace{6pt}	%空格
	
		本文的优点为:1. 采用试探法粗略搜索与二分法精细搜索结合的方法,降低了问题的求解难度。2.以凸轮转动角度为固定步长,对不同角速度按照不同精度的时间步长求解,大大提高了求解的精确度。 3.针对智能算法求解精度方面,采用改进的蝙蝠算法,使速度权重系数自适应调整,兼顾局部搜索与全局搜索能力。
		
		\keywords{
			微分方程\quad
			微分方程\quad	
			微分方程\quad
			微分方程\quad
		}
	\end{abstract}


%%%%%%%%%%%%%%%%%%%%%%%%%%%%%%%%%目录%%%%%%%%%%%%%%%%%%%%%%%%%%%%%%%%%%%%
	\thispagestyle{empty}
	\tableofcontents
	\setcounter{page}{0}                                               
	\newpage	%换页符
	

	
	\section{问题重述}	
		\subsection{问题背景}
	    	新型冠状病毒肺炎(Corona Virus Disease 2019,COVID-19),简称“新冠肺炎”,世界卫生组织命名为“COVID-19”,是指 2019 新型冠状病毒感染导致的肺炎。2020 年 3 月 11 日,世界卫生组织总干事谭德塞宣布,世卫组织认为当前新冠肺炎疫情可被称为全球大流行(pandemic)。目前,COVID-19 疫情仍在世界各地蔓延,已超过 1630 万人感染,65 万余人死亡,给世界各国的经济发展和人民生活带来了极大影响,甚至从一定程度上改变了人类的工作生活方式。
	
		\subsection{问题概述}
		    围绕相关附件和条件要求,定量地研究传染病的传播规律,利用所给(不限于)资料和数据,作出预测并给出控制传染病蔓延的对策建议,具体要求如下:
				 
			
			\textbf{问题一:}
			
			\textbf{问题二:}
			
			\textbf{问题三:}

	
	\section{模型假设}
		\begin{itemize}                                             
		\item [(1)] 
		\item [(2)]
		\item [(3)] 
		\item [(4)] 
		\end{itemize}

		
	\section{符号说明}
		\begin{table}[H]
		\centering
		\setlength{\tabcolsep}{12mm}
		\begin{tabular}{cc}
			\toprule[1.5pt]
			\multicolumn{1}{m{5cm}}{\centering 符号} & \multicolumn{1}{m{5cm}}{\centering 说明} \\
			\midrule[1pt]		
			$P_n$  & 20个站点  \\ 
			$P_n$  & 20个站点  \\ 
		   	$P_n$  & 20个站点  \\ 
			\bottomrule[1.5pt]
		\end{tabular}
		\begin{tablenotes}
		\item 注:表中未说明的符号以首次出现处为准
		\end{tablenotes}
		\end{table}

	\section{问题一模型的建立与求解}
		\subsection{问题描述与分析}
			问题一要求建立确诊和死亡病例数的预测模型,并对具体防控措施进行评价,分析其对疫情传播造成的影响。
			
			经典SEIR模型将人群分为易感者(susceptible, S)、感染者(infected, I)、潜伏者(exposed, E)和康复人群(recovered,R )。该模型还假设人群中所有个体都有被感染的概率,当被感 染个体痊愈后,会产生抗体,即康复人群 R 不会再被感染。考虑到防治传染病的隔离措施, 模型中的人群组别新增隔离易感者(Sq)、隔离潜伏者(Eq)和隔离感染者(Iq)。鉴于隔 离感染者会立即送往定点医院隔离治疗,因此这部分人群在本模型中全部转化为住院患者H。 因此,本文修订的模型中 S、I、E 分别指隔离措施遗漏的易感者、感染者和潜伏者。隔离易 感者解除隔离后重新转变为易感者,而感染者和潜伏者均有不同程度的能力感染易感者,使 其转化为潜伏者。人群的转化关系如图 1 所示。
			
			
			与经典$SIR$模型相比,此次疫情中,不同群体相互转化方式相近,
			在基于人群的方法中,共享相同症状的宿主 按照有限数量的类(也称为仓室)建模或分组,研究人员的 主要任务是研究和比较他们的各种动力学机制。类的组 合用于建模和分析种群动力学。例如,SLIR模型将个人 分为易感、潜伏、感染或恢复四种感染状态之一[10],通过微 分方程来确定流行病学阶段之间的转换。根据被移除的 个体是否会再次变得易感,疾病可以被模拟成SLIR或 SLIR周期。 
			相反,面向网络的方法强调个体的异质性、个体间的 相互作用和网络结构[13]。网络中的个体被表示为节点,他 们之间的交互被表示为链接。网络节点可以用来表示个 人、地点、社区或城市的特征,模型可以结合这些特征的时 间动态,优先定义两个节点之间链接的时间帧,这是一种 常用于表示具有交互或关系模式的个人的组结构的方法[14]。 面向网络的方法适合于捕捉个体间复杂的接触模式,探索 流行病动态和评估公共卫生政策的有效性[15]。格点网络 被用来确定个体之间的距离关系。相比之下,随机网络支 持与移动个人之间的偶然接触相关的特征,以及在社交网 络中常见的低分离度。这些被认为是可靠的调查流行病 的方法,利用特定网络模型的传播动力学来调查新出现的 传染病的传播[16]。近期研究发现,社交网络的拓扑特征对 传染病的传播动力学和临界阈值有很大的影响,从而支持 了面向网络的模型无法进行的细微分析[17]。
		
			其思维流程图如图~\ref{lct}~所示:
			\begin{figure}[H]
				\centering
				\includegraphics[width=\textwidth]{figures/whut.jpg}
				\caption{问题一思维流程图}\label{lct}
			\end{figure}
			
		\subsection{模型的建立}
			本节介绍改进的$SEIR$模型. 研究对象是感染者、潜伏者、易感者、痊愈者等,我们使用如下记号来代表每个人群的人数:
			\begin{itemize}
				\item $S(t)$:$t$时刻易感者的累计总数;
				\item $E(t)$:$t$时刻潜伏者的累计总数;
				\item $I(t)$:$t$时刻感染者的累计总数;
				\item $S_q(t)$:$t$时刻隔离易感者的累计总数;
				\item $E_q(t)$:$t$时刻隔离潜伏者的累计总数;
				\item $H(t)$:$t$时刻住院患者的累计总数;
				\item $R(t)$:$t$时刻痊愈者的累计总数。
			\end{itemize}
		
			模型有以下前提:

			1.潜伏者在出现明显症状前会经历$7$天的潜伏期, 一旦出现症状,潜伏者将寻求治疗,从而转为确诊的感染者;
			
			2.由于政府干预控制措施,部分感染者在潜伏期内尚未出现症状已被隔离,成为隔离潜伏者,在被隔离了平均$14$天后出现症状,成为确诊的感染者。
			
			可建立模型如下:
			
			定义有效接触率$\alpha$、传染率$\beta$和有效接触系数$\rho$,$\alpha$是易感人群在随机混合人群中的占比;$\beta$是有效接触。分析可知易感者$S$有三种转化途径:向隔离易感者$S_q$、隔离潜伏者$E_q$和潜伏者$E$的转化速率(单位时间$\Delta t$内,转化数量$\Delta n$与同类群体的个数$n$的比值)分别为$\rho (1-\beta)q$,$ \rho \beta q$和$\rho \beta(1-q)$。此外,确认隔离期$t_d$无症状后,隔离易感者$S_q$也可向易感者$S$以$\lambda S_{q}$的速率转化。由以上分析可建立易感者$S$转化方程:
			\begin{gather}
			\frac{\mathrm{d} S}{\mathrm{d} t}=-\alpha\rho [\beta +q(1-\beta)](I+\theta E)+\lambda S_{q},
			\end{gather}
			其中,$\theta $是潜伏者相对于感染者传播能力的比值。$\lambda$是隔离解除速率,数值取隔离期的倒数$\lambda=1/14$。
		
			潜伏者可以向感染者转化,易感者可向潜伏者转化,可列出潜伏者$S$的转化方程:
			\begin{gather}
			\frac{\mathrm{d} E}{\mathrm{d} t}=\alpha\rho\beta(1-q) (I+\theta E)-\sigma E,
			\end{gather}
			其中,$\sigma$为潜伏者向感染者的转化速率。
			
			潜伏者可以转化为感染者,感染者的流向有死亡、被治愈和被隔离。定义病死率$d$、感染者恢复率$\varsigma_I$和隔离速率$\sigma_I$,对感染者有:
			\begin{gather}
			\frac{\mathrm{d} I}{\mathrm{d} t}=\sigma E-(\sigma_I+d+\varsigma_I)I.
			\end{gather}
			
			对于被隔离的群体,隔离易感者$S_q$与易感者相互转化;隔离潜伏者$E_q$来源于易感者,可转化为隔离感染者。定义$\delta_q$是隔离潜伏者向隔离感染者的转化速率,则有
			\begin{gather}
			\frac{\mathrm{d} S_q}{\mathrm{d} t}=\alpha\rho q(1-\beta)(I+\theta E)-\lambda S_q,
			\end{gather}
			\begin{gather}
			\frac{\mathrm{d} E_q}{\mathrm{d} t}=\rho \alpha \beta q(I+\theta E)-\delta_q E_q.
			\end{gather}
			
			对住院患者,感染者和隔离的潜伏者向住院患者的转化速率分别是$\delta_I$和$\delta_q$,住院患者流向为死亡和康复,对应系数为死亡率$d$和住院患者恢复率$\varsigma_H$。
			\begin{gather}
			\frac{\mathrm{d}H }{\mathrm{d} t}=\delta_I I+ \delta_q E_q-(d+\varsigma_H )H.
			\end{gather}
			
			最后,痊愈者来源有感染者和住院患者,故对痊愈者有
			\begin{gather}
			\frac{\mathrm{d} R}{\mathrm{d} t}=\varsigma_I I+\varsigma_H H.
			\end{gather}
			
			模型的总表达为:
			\begin{spacing}{2}
			\begin{gather}
			\left\{\begin{array}{l}
			\frac{\mathrm{d} S}{\mathrm{d} t}=-\alpha\rho [\beta +q(1-\beta)](I+\theta E)+\lambda S_{q},
			\\ \frac{\mathrm{d} E}{\mathrm{d} t}=\alpha\rho\beta(1-q) (I+\theta E)-\sigma E,
			\\ \frac{\mathrm{d} I}{\mathrm{d} t}=\sigma E-(\sigma_I+d+\varsigma_I)I,
			\\ \frac{\mathrm{d} S_q}{\mathrm{d} t}=\alpha\rho q(1-\beta)(I+\theta E)-\lambda S_q,
			\\ \frac{\mathrm{d} E_q}{\mathrm{d} t}=\rho \alpha \beta q(I+\theta E)-\delta_q E_q,
			\\ \frac{\mathrm{d}H }{\mathrm{d} t}=\delta_I I+ \delta_q E_q-(d+\varsigma_H )H,
			\\ \frac{\mathrm{d} R}{\mathrm{d} t}=\varsigma_I I+\varsigma_H H.
			\end{array}\right.
			\end{gather}
			\end{spacing}
		\subsection{模型的求解}
			\begin{algorithm}[H]
			 	\caption{Procedure of Apriori}  
			 	\LinesNumbered  
			 	\setstretch{0.9}   %设置表的行间距
			 	\KwIn{item data base: $D$\newline
			 		minimum Support threshold: $Sup_{min}$\newline
			 		minimum Confidence threshold: $Conf_{min}$
			 	}
			 	\KwOut{frequent item sets $F$}  
			 	\textbf{Initialize} \newline
			 	iteration $t\leftarrow 1$ \newline
			 	The candidate FIS:$C_{t}=\varnothing$ \newline
			 	The length of FIS:$length=1$ \newline
			 	\For{i=1 to sizeof(D)}
			 	{$I_{i}$=D(i)\newline
			 		n=sizeof($I_{i}$)\newline
			 		\For{j=1 to n}{
			 			\If{$I_{i}(j)\notin C_{t}$ }
			 			{$C_{t}=C_{t}\cup I_{i}(j) $}
			 		}
			 	}
			 	$F_{t}=\left \{ f|f\in C_{t},Sup(f)>Sup_{min}\right \}$\newline
			 	\While{$F\neq \varnothing$}
			 	{ t=t+1\newline 
			 		length=length+1\newline	
			 		$C_{t}\leftarrow $ all candidate of FIS in $F_{t-1}$\newline
			 		$F_{t}=\left \{ f|f\in C_{t},(Sup(f)>Sup_{min})\bigcap (Comf(f)>Conf_{min}) \right\}$\newline
			 	}	
			 	\Return{$F_{t-1}$} 
			\end{algorithm} 
		
        \subsection{实验结果及分析}
  
	\section{问题二模型的建立与求解}
		\subsection{问题描述与分析}
			问题二要求

    		其思维流程图如图~\ref{lssssct}~所示:

			\begin{figure}[H]
				\centering
				\includegraphics[width=\textwidth]{figures/whut.jpg}
				\caption{问题二思维流程图}\label{lssssct}
			\end{figure}

		\subsection{模型的建立}
		
		\subsection{模型的求解}

        \subsection{实验结果及分析}
        
			结果如下表\ref{zhuanssssasgzai}所示:
			\begin{table}[H]
			\setstretch{1.4}  %设置表的行间距
			\centering		
			\caption{xxxxxxxxxxxxxxxxxxxxx}\label{biao1}
			\begin{tabular}{cc}
			\toprule[2pt]
				\multicolumn{1}{m{5cm}}{\centering xxxxxxx}
				& \multicolumn{1}{m{5cm}}{\centering xxxxxxx}
				\\
				\midrule[1pt]
				xxxxxxx &   909.80\\ 
				xxxxxxx & 	852.60\\ 
			\bottomrule[2pt]	
			\end{tabular}
			\end{table}
  
  			由表\ref{biao1}可知

			其各个小车的运输细节图下图所示:
			\begin{figure}[H]
				\centering
				\subfigure{\includegraphics[height=8cm,width=7.5cm]{figures/whut.jpg}}
				\subfigure{\includegraphics[height=8cm,width=7.5cm]{figures/whut.jpg}}
			\end{figure}	
			\begin{figure}[H]	
				\centering
				\subfigure{\includegraphics[height=8cm,width=7.5cm]{figures/whut.jpg}}
				\subfigure{\includegraphics[height=8cm,width=7.5cm]{figures/whut.jpg}}
				\caption{xxxxxxxxxxxxxxxxxxxxxxxxx}
				\label{fisg}
			\end{figure}

    \section{问题三模型的建立与求解}
  		\subsection{结果分析}
  
  	\section{灵敏度分析}
 
  	\section{模型的评价}
		\subsection{模型的优点}
			\begin{itemize}                                             
			\item [(1)]
			\item [(2)] 	
			\end{itemize}
		\subsection{模型的缺点}

  		\subsection{模型改进}

  
  
 
	\newpage	%换页符
	%%参考文献
	%\begin{thebibliography}{9}%宽度9
	% \setlength{\itemsep}{-2mm}
	\nocite{*}		%排版未引用的参考文献
	\begin{thebibliography}{9}%宽度9
		\bibitem{1}张斯嘉, 郭建胜, 钟夫, 等. 基于蝙蝠算法的多目标战备物资调运决策优化[J]. 火力与指挥控制, 2016, 41(1): 58-61.
	
	\end{thebibliography}

	\newpage
	%附录
	\appendix %%附录
	\section{数据可视化的实现}
		\subsection*{第一问画图--python源代码}
			\begin{lstlisting}[language=python]
			
			\end{lstlisting}
			
		\subsection*{第二问画图--python源代码}
			\lstinputlisting[language={python},numbers=left,numberstyle=\tiny,
			rulesepcolor=\color{red!20!green!20!blue!20},  
			keywordstyle=\color{blue!70!black},  
			commentstyle=\color{blue!90!},  
			basicstyle=\ttfamily] {./code/demo.py}

\end{document}
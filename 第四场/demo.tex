\documentclass{whutmod}
\usepackage[linesnumbered,ruled,lined]{algorithm2e}
\bibliographystyle{unsrt}
\team{10}
\membera{刘子川}
\joba{编程}
\memberb{程宇}
\jobb{建模}
\memberc{祁成}
\jobc{写作}
\hypersetup{
	colorlinks=true,
	linkcolor=black,citecolor=black
}


\newcommand{\upcite}[1]{\textsuperscript{\cite{#1}}}
%%%%%%%%%%%%%%%%%%%%%%%%%%%%%%%%%题目%%%%%%%%%%%%%%%%%%%%%%%%%%%%%%%%%%%%
\title{基于xxxxxxxx模型}
\tihao{1} 
\everymath{\displaystyle} 
\begin{document}

	\maketitle
	\thispagestyle{empty}
%%%%%%%%%%%%%%%%%%%%%%%%%%%%%%%%%摘要%%%%%%%%%%%%%%%%%%%%%%%%%%%%%%%%%%%%
	\begin{abstract}
		控制高压油管的压力变化对减小燃油量偏差,提高发动机工作效率具有重要意义。本文建立了基于质量守恒定理的微分方程稳压模型,采用二分法、试探法以及自适应权重的蝙蝠算法对模型进行求解。
		\vspace{6pt}	%空格


		针对问题一,建立基于质量守恒定律的燃油流动模型
		
		%针对问题一,建立基于质量守恒定律的燃油流动模型,考察单向阀开启时间对压力稳定性的影响。综合考虑压力与弹性模量、密度之间的关系,提出燃油压力-密度微分方程模型和燃油流动方程。本文采用改进的欧拉方法对燃油压力-密度微分方程求得数值解;利用二分法求解压力分布。综合考虑平均绝对偏差等反映压力稳定程度的统计量,求得直接稳定于100MPa的开启时长为\textbf{0.2955ms} ,在2s、5s内到达并稳定于150MPa时开启时长为\textbf{0.7795ms}、\textbf{0.6734ms},10s到达并稳定于150MPa的开启时长存在多解。最后对求解结果进行灵敏度分析、误差分析。
		\vspace{6pt}	%空格
		

	
		针对问题二,建立基于质量守恒定律的泵-管-嘴系统动态稳压模型,将燃油进入和喷出的过程动态化处理。考虑柱塞和针阀升程的动态变动,建立喷油嘴流量方程和质量守恒方程。为提高角速度求解精度,以凸轮转动角度为固定步长,转动时间变动步长,采用试探法粗略搜索与二分法精细搜索的方法求解,求得凸轮最优转动角速度\textbf{0.0283rad/ms(转速270.382转/分钟)},并得到该角速度下高压油管的密度、压力周期性变化图。对求解结果进行误差分析与灵敏度分析,考察柱塞腔残余容积变动对高压油管压力稳态的影响。
		\vspace{6pt}	%空格
	
		针对问题三,对于增加一个喷油嘴的情况,改变质量守恒方程并沿用问题二的模型调整供、喷油策略,得到最优凸轮转动角速度为\textbf{0.0522rad/ms(498.726转/分钟)};对于既增加喷油嘴又增加减压阀的情况,建立基于自适应权重的蝙蝠算法的多变量优化模型,以凸轮转动角速度、减压阀开启时长和关闭时长为参数,平均绝对偏差MAD为目标,在泵-管-嘴系统动态稳压模型的基础上进行求解,得到最优参数:\textbf{角速度0.0648 rad/ms(619.109转/分钟)}、减压阀的开启时长\textbf{2.4ms}和减压阀的关闭时长\textbf{97.6ms}。
		\vspace{6pt}	%空格
	
		本文的优点为:1. 采用试探法粗略搜索与二分法精细搜索结合的方法,降低了问题的求解难度。2.以凸轮转动角度为固定步长,对不同角速度按照不同精度的时间步长求解,大大提高了求解的精确度。 3.针对智能算法求解精度方面,采用改进的蝙蝠算法,使速度权重系数自适应调整,兼顾局部搜索与全局搜索能力。
		
		\keywords{
			微分方程\quad
			微分方程\quad	
			微分方程\quad
			微分方程\quad
		}
	\end{abstract}


%%%%%%%%%%%%%%%%%%%%%%%%%%%%%%%%%目录%%%%%%%%%%%%%%%%%%%%%%%%%%%%%%%%%%%%
	\thispagestyle{empty}
	\tableofcontents
	\setcounter{page}{0}                                               
	\newpage	%换页符
	

	
	\section{问题重述}	
		\subsection{问题背景}
	    	新型冠状病毒肺炎(Corona Virus Disease 2019,COVID-19),简称“新冠肺炎”,世界卫生组织命名为“COVID-19”,是指 2019 新型冠状病毒感染导致的肺炎。2020 年 3 月 11 日,世界卫生组织总干事谭德塞宣布,世卫组织认为当前新冠肺炎疫情可被称为全球大流行(pandemic)。目前,COVID-19 疫情仍在世界各地蔓延,已超过 1630 万人感染,65 万余人死亡,给世界各国的经济发展和人民生活带来了极大影响,甚至从一定程度上改变了人类的工作生活方式。
	
		\subsection{问题概述}
		    围绕相关附件和条件要求,定量地研究传染病的传播规律,利用所给(不限于)资料和数据,作出预测并给出控制传染病蔓延的对策建议,具体要求如下:
				 
			
			\textbf{问题一:}
			
			\textbf{问题二:}
			
			\textbf{问题三:}

	
	\section{模型假设}
		\begin{itemize}                                             
		\item [(1)] 
		\item [(2)]
		\item [(3)] 
		\item [(4)] 
		\end{itemize}

		
	\section{符号说明}
		\begin{table}[H]
		\centering
		\setlength{\tabcolsep}{12mm}
		\begin{tabular}{cc}
			\toprule[1.5pt]
			\multicolumn{1}{m{5cm}}{\centering 符号} & \multicolumn{1}{m{5cm}}{\centering 说明} \\
			\midrule[1pt]		
			$P_n$  & 20个站点  \\ 
			$P_n$  & 20个站点  \\ 
		   	$P_n$  & 20个站点  \\ 
			\bottomrule[1.5pt]
		\end{tabular}
		\begin{tablenotes}
		\item 注:表中未说明的符号以首次出现处为准
		\end{tablenotes}
		\end{table}

	\section{问题一模型的建立与求解}
		\subsection{问题描述与分析}
			问题一要求建立确诊和死亡病例数的预测模型,并对具体防控措施进行评价,分析其对疫情传播造成的影响。
			
			此次疫情中,不同群体相互转化方式相近。目前对传染病的模型建立方法主要有面向网络的方法和基于人群的方法。面向网络的方法强调个体的异质性、个体间的相互作用和网络结构[13]。利用特定网络模型的传播动力学来调查新出现的 传染病的传播[16]。近期研究发现,社交网络的拓扑特征对传染病的传播动力学和临界阈值有很大的影响,从而支持了面向网络的模型无法进行的细微分析[17]。相反,基于人群的方法通过微分方程比较他们的各种动力学机制,确定流行病学阶段之间的转换。根据被移除的个体是否会再次变得易感,疾病可以被模拟成SLIR或SLIR周期。
			
			经典$SEIR$模型将人群分为易感者$(susceptible,S)$、感染者$(infected,I)$、潜伏者$(exposed,E)$和康复人群$(recovered,R)$。模型假设所有易感者都有概率转变为感染者;潜伏者是具有时滞效应的感染者,在经过无症状的潜伏期后转变为感染者;部分感染者可被治愈,成为痊愈者。考虑政府干预,为便于进一步分析政府宏观措施带来的防疫效应,在经典SEIR模型的基础上增加人群种类:隔离易感者$(quarantined\ susceptible, S_q)$,隔离潜伏者$(quarantined\ exposed, E_q)$和住院患者$(hospitalized,H)$。隔离易感者解除隔离后重新转变为易感者,感染者和潜伏者均有不同程度的能力感染易感者,使易感者成为潜伏者或感染者。人群的转化关系如图所示:
			\begin{figure}[H]
				\centering
				\includegraphics[width=0.8\textwidth]{figures/kkkk.png}
				\caption{问题一人群转化关系图}\label{lct}
			\end{figure}
			
		\subsection{模型的建立}
			本节介绍改进的$SEIR$模型. 研究对象是感染者、潜伏者、易感者、痊愈者等,我们使用如下记号来代表每个人群的人数:
			\begin{itemize}
				\item $S(t)$:$t$时刻易感者的累计总数;
				\item $E(t)$:$t$时刻潜伏者的累计总数;
				\item $I(t)$:$t$时刻感染者的累计总数;
				\item $S_q(t)$:$t$时刻隔离易感者的累计总数;
				\item $E_q(t)$:$t$时刻隔离潜伏者的累计总数;
				\item $H(t)$:$t$时刻住院患者的累计总数;
				\item $R(t)$:$t$时刻痊愈者的累计总数。
			\end{itemize}
		
			模型有以下前提:

			1.潜伏者在出现明显症状前会经历$7$天的潜伏期, 一旦出现症状,潜伏者将寻求治疗,从而转为确诊的感染者;
			
			2.由于政府干预控制措施,部分感染者在潜伏期内尚未出现症状已被隔离,成为隔离潜伏者,在被隔离了平均$14$天后出现症状,成为确诊的感染者。
			
			可建立模型如下:
			
			定义有效接触率$\alpha$、传染率$\beta$和有效接触系数$\rho$,$\alpha$是易感人群在随机混合人群中的占比;$\beta$是有效接触。分析可知易感者$S$有三种转化途径:向隔离易感者$S_q$、隔离潜伏者$E_q$和潜伏者$E$的转化速率(单位时间$\Delta t$内,转化数量$\Delta n$与同类群体的个数$n$的比值)分别为$\rho (1-\beta)q$,$ \rho \beta q$和$\rho \beta(1-q)$。此外,确认隔离期$t_d$无症状后,隔离易感者$S_q$也可向易感者$S$以$\lambda S_{q}$的速率转化。由以上分析可建立易感者$S$转化方程:
			\begin{gather}
			\frac{\mathrm{d} S}{\mathrm{d} t}=-\alpha\rho [\beta +q(1-\beta)]S(I+\theta E)+\lambda S_{q},
			\end{gather}
			其中,$\theta $是潜伏者相对于感染者传播能力的比值。$\lambda=1/14$是隔离解除速率,数值取隔离期的倒数。
		
			潜伏者可以向感染者转化,易感者可向潜伏者转化,可列出潜伏者$S$的转化方程:
			\begin{gather}
			\frac{\mathrm{d} E}{\mathrm{d} t}=\alpha\rho\beta(1-q) (I+\theta E)-\sigma E,
			\end{gather}
			其中,$\sigma$为潜伏者向感染者的转化速率。
			
			潜伏者可以转化为感染者,感染者的流向有死亡、被治愈和被隔离。定义病死率$d$、感染者恢复率$\varsigma_I$和隔离速率$\sigma_I$,对感染者有:
			\begin{gather}
			\frac{\mathrm{d} I}{\mathrm{d} t}=\sigma E-(\sigma_I+d+\varsigma_I)I.
			\end{gather}
			
			对于被隔离的群体,隔离易感者$S_q$与易感者相互转化;隔离潜伏者$E_q$来源于易感者,可转化为隔离感染者。定义$\delta_q$是隔离潜伏者向隔离感染者的转化速率,则有
			\begin{gather}
			\frac{\mathrm{d} S_q}{\mathrm{d} t}=\alpha\rho q(1-\beta)(I+\theta E)-\lambda S_q,
			\end{gather}
			\begin{gather}
			\frac{\mathrm{d} E_q}{\mathrm{d} t}=\rho \alpha \beta q(I+\theta E)-\delta_q E_q.
			\end{gather}
			
			对住院患者,感染者和隔离的潜伏者向住院患者的转化速率分别是$\delta_I$和$\delta_q$,住院患者流向为死亡和康复,对应系数为死亡率$d$和住院患者恢复率$\varsigma_H$。
			\begin{gather}
			\frac{\mathrm{d}H }{\mathrm{d} t}=\delta_I I+ \delta_q E_q-(d+\varsigma_H )H.
			\end{gather}
			
			对死亡病例,有
			\begin{gather}
			\frac{\mathrm{d} D}{\mathrm{d} t}= D + d(H+I).
			\end{gather}
			
			最后,痊愈者来源有感染者和住院患者,故对痊愈者有
			\begin{gather}
			\frac{\mathrm{d} R}{\mathrm{d} t}=\varsigma_I I+\varsigma_H H.
			\end{gather}
			
			模型的总表达为:
			\begin{spacing}{2}
			\begin{gather}
			\left\{\begin{array}{l}
			\frac{\mathrm{d} S}{\mathrm{d} t}=-\alpha\rho [\beta +q(1-\beta)]S(I+\theta E)+\lambda S_{q},
			\\ \frac{\mathrm{d} E}{\mathrm{d} t}=\alpha\rho\beta(1-q) (I+\theta E)-\sigma E,
			\\ \frac{\mathrm{d} I}{\mathrm{d} t}=\sigma E-(\sigma_I+d+\varsigma_I)I,
			\\ \frac{\mathrm{d} S_q}{\mathrm{d} t}=\alpha\rho q(1-\beta)(I+\theta E)-\lambda S_q,
			\\ \frac{\mathrm{d} E_q}{\mathrm{d} t}=\rho \alpha \beta q(I+\theta E)-\delta_q E_q,
			\\ \frac{\mathrm{d}H }{\mathrm{d} t}=\delta_I I+ \delta_q E_q-(d+\varsigma_H )H,
			\\ \frac{\mathrm{d} R}{\mathrm{d} t}=\varsigma_I I+\varsigma_H H,
			\\\frac{\mathrm{d} D}{\mathrm{d} t}= D + d(H+I).
			\end{array}\right.
			\end{gather}
			\end{spacing}
		\subsection{模型的求解}
			\begin{algorithm}[H]
			 	\caption{Procedure of Apriori}  
			 	\LinesNumbered  
			 	\setstretch{0.9}   %设置表的行间距
			 	\KwIn{item data base: $D$\newline
			 		minimum Support threshold: $Sup_{min}$\newline
			 		minimum Confidence threshold: $Conf_{min}$
			 	}
			 	\KwOut{frequent item sets $F$}  
			 	\textbf{Initialize} \newline
			 	iteration $t\leftarrow 1$ \newline
			 	The candidate FIS:$C_{t}=\varnothing$ \newline
			 	The length of FIS:$length=1$ \newline
			 	\For{i=1 to sizeof(D)}
			 	{$I_{i}$=D(i)\newline
			 		n=sizeof($I_{i}$)\newline
			 		\For{j=1 to n}{
			 			\If{$I_{i}(j)\notin C_{t}$ }
			 			{$C_{t}=C_{t}\cup I_{i}(j) $}
			 		}
			 	}
			 	$F_{t}=\left \{ f|f\in C_{t},Sup(f)>Sup_{min}\right \}$\newline
			 	\While{$F\neq \varnothing$}
			 	{ t=t+1\newline 
			 		length=length+1\newline	
			 		$C_{t}\leftarrow $ all candidate of FIS in $F_{t-1}$\newline
			 		$F_{t}=\left \{ f|f\in C_{t},(Sup(f)>Sup_{min})\bigcap (Comf(f)>Conf_{min}) \right\}$\newline
			 	}	
			 	\Return{$F_{t-1}$} 
			\end{algorithm} 
=======
			问题一要求
		
			其思维流程图如图~\ref{lct}~所示:
			\begin{figure}[H]
				\centering
				\includegraphics[width=\textwidth]{figures/whut.jpg}
				\caption{问题一思维流程图}\label{lct}
			\end{figure}
			
		\subsection{模型的建立}
			\begin{gather*}
			d(p_i,p_j)=\left | x_i-x_j \right |+\left | y_i-y_j \right | ,
			\end{gather*}
		
		
		\subsection{模型的求解}
		\subsubsection{免疫差分进化算法}
		%\paragraph{初始化}
		本文设计免疫差分进化算法估计微分方程组中的未知参数,定义决策向量为
		\begin{gather}
		X=[x_1,x_2,x_3,x_4],
		\end{gather}
		其中$x_1$、$x_2$、$x_3$和$x_4$分别表示每个患者的日平均接触人数、传染概率、初始潜伏者与患者人数比例以及潜伏者相对于感染者传播能力的比值。将目标函数定义为损失函数如下
		\begin{gather}
		min Loss(X)=\sum_{t=1}^T (|\frac{D_r(t)-D(t)}{D_r(t)}|+|\frac{R_r(t)-R(t)}{R_r(t)}|+|\frac{H_r(t)-H(t)}{H_r(t)}|)
		\end{gather}
		其中$T$表示选取数据的终止节点,即表示选取用于估计参数的数据来自疫情发生的第$1$天到第$T$天。$D_r(t)$、$R_r(t)$与$H_r(t)$分别表示疫情发生后第$t$天的死亡人数、治愈人数和医院患者人数的真实数据;$D(t)$、$R(t)$与$H(t)$分别表示其对应的由SEIR模型。损失函数$Loss$表示预测结果与实际结果间的距离,即$Loss$值越小,预测曲线就与真实曲线越接近。
		\paragraph{种群初始化}
		在解空间中随机产$p$个初始个体$
		X_i(0)=[x_1,x_2,x_3,x_4],(i=1,2,3,\cdots,p).
		$
		其中第$i$个个体的第$j$维取值方式如下
		\begin{gather}
		x_{i,j}(0)=x_{j,min}+rand(0,1)(x_{j,max}-x_{j,min}),\\i=1,2,3,\cdots,p,j=1,2,3,4
		\end{gather}
		其中$p$表示种群规模,$x_{j,max}$和$x_{j,min}$分别表示决策变量$X$第$j$维的
		取值范围上界与下界。
		\paragraph{变异}
		在第$g$次迭代中,生成变异个体$H_i(g)$,从种群中随机选取三个个体$X_{p1}(g)$,$X_{p2}(g)$和$X_{p3}(g)$,且$p_1\neq p_2\neq p_3\neq i$,生成的变异向量为
	    \begin{gather}
	    H_i(g)=X_{p1}(g)+F(g)*(X_{p2}(g)-X_{p3}(g))
	    \end{gather}
	    $F(g)\in (0,1)$是每一代中的放缩因子,其服从柯西分部如下
	    \begin{gather}
	    F(g)=cauchyrnd(uF,0.1)
	    \end{gather}
	   其中$uF$是$F$的期望值,本文取值为$uF=0.5$。
		\paragraph{交叉}
		对第$g$代种群中第$i$个体进行交叉操作,生成交叉个体$V_i(g)$,具体表达式如下:
		\begin{gather}
		v_{i,j}=\left\{\begin{matrix}h_{i,j}(g),rand(0,1)\leq cr_{i}
		\\ x_{i,j}(g),rand(0,1)>cr_{i}
		\end{matrix}\right.
		\end{gather}
     	其中$cr_{i}\in[0.1,0.6]$是个体$i$的交叉概率,参数$cr_{i}$将进行自适应调整,具体表达式如下:
		\begin{gather}
		cr_{i}=\left\{\begin{matrix}cr_{l}+(cr_{u}-cr_{l})\frac{Loss_{i}-Loss_{min}}{Loss_{max}-Loss_{min}} , Loss_{i}>\overline{Loss}
		\\ cr_{l},Loss_{i}\leqslant  \overline{Loss}
		\end{matrix}\right.
		\end{gather}
	    \paragraph{免疫选择}
	    混合第$g$代的交叉个体$V(g)$与原始个体$X(g)$,得到待选组$\left \{ X '(g+1)\right \}$如下
	    \begin{gather*}
	    X_i '(g+1)=\left\{\begin{matrix}  X_i (g),i\leqslant p
	    \\  V_{i-p} (g),i>p.
	    \end{matrix}\right.
	    \end{gather*}
		个体 $X_a '(g+1)$和$X_b '(g+1)$的亲和度$S_{a,b}$可表示为
		\begin{gather}
		S_{a,b}=\sqrt{\sum _{i=1}^4( \frac{x_{i,a}-x_{i,b}}{x_{i,max}-x_{i,min}})^2}
		\end{gather}
		$S_{a,b}$为$X_a '(g+1)$和$X_b '(g+1)$的归一化距离,表示个体$X_a '(g+1)$和$X_b '(g+1)$的相似性。定义个体$X_i '(g+1)$的抗体浓度为$C_{i}$,即
		\begin{gather}
		C_{i}=\frac{1}{2p}\sum _{j=1}^{2p} N_{i,j},\\
		N_{i,j}=\left\{\begin{matrix}1,S_{i,j}\geqslant \mu 
		\\ 0,S_{i,j}< \mu 
		\end{matrix}\right.
		\end{gather}
		$\mu(\mu\in[0,1])$为相似度阈值,即当个体$i$和$j$的亲和度$S_{i,j}\geqslant \mu$时认为个体$i$和$j$为相似个体。$C_{i}$即为$\left \{ X '(g+1)\right \}$中$X_i '(g+1)$的相似个体所占比例,$C_{i}$越大即表示$X_i '(g+1)$所在区域的个体密度越大。我们优先将损失函数$Loss$值最优的前$\sigma$个解放入下一代个体$\left \{ X(g+1)\right \}$中以防止最优解丢失。再计算剩余个体的复合适应度函数,即个体$i$的复合适应度函数可表示为
	    \begin{gather}
	  min F(X_i '(g+1))=\frac{Loss(X_i '(g+1))-Loss_{min}}{Loss_{max}-loss_{min}}+C_{i}
		\end{gather}
		即选取复合适应度函数$F$较优的剩余$p-\sigma$个个体放入下一代个体$\left \{ X(g+1)\right \}$中。重复迭代上述算法$G$次后终止算法并输出最优参数集$X_{best}$。
		
		
		
%				\begin{algorithm}[H]
%			 	\caption{Procedure of Apriori}  
%			 	\LinesNumbered  
%			 	\setstretch{0.9}   %设置表的行间距
%			 	\KwIn{item data base: $D$\newline
%			 		minimum Support threshold: $Sup_{min}$\newline
%			 		minimum Confidence threshold: $Conf_{min}$
%			 	}
%			 	\KwOut{frequent item sets $F$}  
%			 	\textbf{Initialize} \newline
%			 	iteration $t\leftarrow 1$ \newline
%			 	The candidate FIS:$C_{t}=\varnothing$ \newline
%			 	The length of FIS:$length=1$ \newline
%			 	\For{i=1 to sizeof(D)}
%			 	{$I_{i}$=D(i)\newline
%			 		n=sizeof($I_{i}$)\newline
%			 		\For{j=1 to n}{
%			 			\If{$I_{i}(j)\notin C_{t}$ }
%			 			{$C_{t}=C_{t}\cup I_{i}(j) $}
%			 		}
%			 	}
%			 	$F_{t}=\left \{ f|f\in C_{t},Sup(f)>Sup_{min}\right \}$\newline
%			 	\While{$F\neq \varnothing$}
%			 	{ t=t+1\newline 
%			 		length=length+1\newline	
%			 		$C_{t}\leftarrow $ all candidate of FIS in $F_{t-1}$\newline
%			 		$F_{t}=\left \{ f|f\in C_{t},(Sup(f)>Sup_{min})\bigcap (Comf(f)>Conf_{min}) \right\}$\newline
%			 	}	
%			 	\Return{$F_{t-1}$} 
%			\end{algorithm} 

		
        \subsection{实验结果及分析}
  
	\section{问题二模型的建立与求解}
		\subsection{问题描述与分析}
			问题二要求

    		其思维流程图如图~\ref{lssssct}~所示:

			\begin{figure}[H]
				\centering
				\includegraphics[width=\textwidth]{figures/whut.jpg}
				\caption{问题二思维流程图}\label{lssssct}
			\end{figure}

		\subsection{模型的建立}
		
		\subsection{模型的求解}

        \subsection{实验结果及分析}
        
			结果如下表\ref{zhuanssssasgzai}所示:
			\begin{table}[H]
			\setstretch{1.4}  %设置表的行间距
			\centering		
			\caption{xxxxxxxxxxxxxxxxxxxxx}\label{biao1}
			\begin{tabular}{cc}
			\toprule[2pt]
				\multicolumn{1}{m{5cm}}{\centering xxxxxxx}
				& \multicolumn{1}{m{5cm}}{\centering xxxxxxx}
				\\
				\midrule[1pt]
				xxxxxxx &   909.80\\ 
				xxxxxxx & 	852.60\\ 
			\bottomrule[2pt]	
			\end{tabular}
			\end{table}
  
  			由表\ref{biao1}可知

			其各个小车的运输细节图下图所示:
			\begin{figure}[H]
				\centering
				\subfigure{\includegraphics[height=8cm,width=7.5cm]{figures/whut.jpg}}
				\subfigure{\includegraphics[height=8cm,width=7.5cm]{figures/whut.jpg}}
			\end{figure}	
			\begin{figure}[H]	
				\centering
				\subfigure{\includegraphics[height=8cm,width=7.5cm]{figures/whut.jpg}}
				\subfigure{\includegraphics[height=8cm,width=7.5cm]{figures/whut.jpg}}
				\caption{xxxxxxxxxxxxxxxxxxxxxxxxx}
				\label{fisg}
			\end{figure}

    \section{问题三模型的建立与求解}
  		\subsection{结果分析}
  
  	\section{灵敏度分析}
 
  	\section{模型的评价}
		\subsection{模型的优点}
			\begin{itemize}                                             
			\item [(1)]
			\item [(2)] 	
			\end{itemize}
		\subsection{模型的缺点}

  		\subsection{模型改进}

  
  
 
	\newpage	%换页符
	%%参考文献
	%\begin{thebibliography}{9}%宽度9
	% \setlength{\itemsep}{-2mm}
	\nocite{*}		%排版未引用的参考文献
	\begin{thebibliography}{9}%宽度9
		\bibitem{1}张斯嘉, 郭建胜, 钟夫, 等. 基于蝙蝠算法的多目标战备物资调运决策优化[J]. 火力与指挥控制, 2016, 41(1): 58-61.
	
	\end{thebibliography}

	\newpage
	%附录
	\appendix %%附录
	\section{数据可视化的实现}
		\subsection*{第一问画图--python源代码}
			\begin{lstlisting}[language=python]
			
			\end{lstlisting}
			
		\subsection*{第二问画图--python源代码}
			\lstinputlisting[language={python},numbers=left,numberstyle=\tiny,
			rulesepcolor=\color{red!20!green!20!blue!20},  
			keywordstyle=\color{blue!70!black},  
			commentstyle=\color{blue!90!},  
			basicstyle=\ttfamily] {./code/demo.py}

\end{document}
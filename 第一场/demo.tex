\documentclass{whutmod}
\usepackage[linesnumbered,ruled,lined]{algorithm2e}
\bibliographystyle{unsrt}
\team{A010}
\membera{刘子川}
\joba{编程}
\memberb{程宇}
\jobb{建模}
\memberc{祁成}
\jobc{写作}
\hypersetup{
	colorlinks=true,
	linkcolor=black
}


\newcommand{\upcite}[1]{\textsuperscript{\cite{#1}}}
%%%%%%%%%%%%%%%%%%%%%%%%%%%%%%%%%题目%%%%%%%%%%%%%%%%%%%%%%%%%%%%%%%%%%%%
\title{基于xxxxxxxx模型}
\tihao{1} 

\begin{document}

	\maketitle
	\thispagestyle{empty}
%%%%%%%%%%%%%%%%%%%%%%%%%%%%%%%%%摘要%%%%%%%%%%%%%%%%%%%%%%%%%%%%%%%%%%%%
	\begin{abstract}
		控制高压油管的压力变化对减小燃油量偏差,提高发动机工作效率具有重要意义。本文建立了基于质量守恒定理的微分方程稳压模型,采用二分法、试探法以及自适应权重的蝙蝠算法对模型进行求解。
		//
	
		针对问题一,建立基于质量守恒定律的燃油流动模型,考察单向阀开启时间对压力稳定性的影响。综合考虑压力与弹性模量、密度之间的关系,提出燃油压力-密度微分方程模型和燃油流动方程。本文采用改进的欧拉方法对燃油压力-密度微分方程求得数值解;利用二分法求解压力分布。综合考虑平均绝对偏差等反映压力稳定程度的统计量,求得直接稳定于100MPa的开启时长为\textbf{0.2955ms} ,在2s、5s内到达并稳定于150MPa时开启时长为\textbf{0.7795ms}、\textbf{0.6734ms},10s到达并稳定于150MPa的开启时长存在多解。最后对求解结果进行灵敏度分析、误差分析。
		//
	
		针对问题二,建立基于质量守恒定律的泵-管-嘴系统动态稳压模型,将燃油进入和喷出的过程动态化处理。考虑柱塞和针阀升程的动态变动,建立喷油嘴流量方程和质量守恒方程。为提高角速度求解精度,以凸轮转动角度为固定步长,转动时间变动步长,采用试探法粗略搜索与二分法精细搜索的方法求解,求得凸轮最优转动角速度\textbf{0.0283rad/ms(转速270.382转/分钟)},并得到该角速度下高压油管的密度、压力周期性变化图。对求解结果进行误差分析与灵敏度分析,考察柱塞腔残余容积变动对高压油管压力稳态的影响。
		//
	
		针对问题三,对于增加一个喷油嘴的情况,改变质量守恒方程并沿用问题二的模型调整供、喷油策略,得到最优凸轮转动角速度为\textbf{0.0522rad/ms(498.726转/分钟)};对于既增加喷油嘴又增加减压阀的情况,建立基于自适应权重的蝙蝠算法的多变量优化模型,以凸轮转动角速度、减压阀开启时长和关闭时长为参数,平均绝对偏差MAD为目标,在泵-管-嘴系统动态稳压模型的基础上进行求解,得到最优参数:\textbf{角速度0.0648 rad/ms(619.109转/分钟)}、减压阀的开启时长\textbf{2.4ms}和减压阀的关闭时长\textbf{97.6ms}。
		//
	
		本文的优点为:1. 采用试探法粗略搜索与二分法精细搜索结合的方法,降低了问题的求解难度。2.以凸轮转动角度为固定步长,对不同角速度按照不同精度的时间步长求解,大大提高了求解的精确度。 3.针对智能算法求解精度方面,采用改进的蝙蝠算法,使速度权重系数自适应调整,兼顾局部搜索与全局搜索能力。
		
		\keywords{
			微分方程\quad
			微分方程\quad	
			微分方程\quad
			微分方程\quad
		}
	\end{abstract}


%%%%%%%%%%%%%%%%%%%%%%%%%%%%%%%%%目录%%%%%%%%%%%%%%%%%%%%%%%%%%%%%%%%%%%%
	\thispagestyle{empty}
	\tableofcontents
	\setcounter{page}{0}                                               
	\newpage	%换页符
	

	
	\section{问题重述}	
		\subsection{问题背景}
	    	分析研究\upcite{1}。xxxxxxxxxxx\footnote{\quad xxxxxxxxxx.}.
	村通自来水工程是指在现有农村居民饮水安全工程的基础上,通过扩网、改造、联通、整合和新建等措施,把符合国家水质标准的自来水引接到行政村和有条件的自然 村,形成具有高保证率和统一供水标准的农村供水网络,基本形成覆盖全县农村的供水安全保障体系,实现农村供水由点到面、由小型分散供水到适度集中供水、由 解决水量及常规水质到水量、水质、水压达标等方面的提升,使广大农村居民长期受益,实现我县农村饮水“提质增效升级”的目的。
	
	自来水管道铺设是搭建自来水系统的重要环节,合理的管道铺设方案可以大幅度节约成本。本问题要求在充分考虑市场因素后,研究用两种不同型号的管道铺设该村的自来水管道的方案,使得建设成本降低。由于不同类形的管道的成本不同,且在实际应用中自来水厂有功率限制,研究自来水管的铺设对于村通自来水工程有着重要意义。
	
		\subsection{问题概述}
		    围绕相关附件和条件要求,研究两种型号的管道在各自来水厂间的铺设方案,依次提出以下问题:
		    
			\textbf{问题一:}
			设计从中心供水站A出发使得自来水管道的总里程最少的铺设方案,并求出该方案下I型管道和II型管道总里程数。
			
			\textbf{问题二:}
			由于二型管道数量不足,设计自来水厂升级方案使得两个二级自来水厂升级为一级自来水厂,使得二级管道的使用量尽可能减小。
			
			\textbf{问题三:}
            考虑自来水厂的功率限制,设计升级方案使得若干的二级自来水厂升级为一级,并求解该情况下的最小铺设总长度。
	
	\section{模型假设}
		\begin{itemize}                                             
		\item [(1)] 
		\item [(2)]
		\item [(3)] 
		\item [(4)] 
		\end{itemize}

		
	\section{符号说明}
		\begin{table}[H]
		\centering
		\setlength{\tabcolsep}{12mm}
		\begin{tabular}{cc}
			\toprule[1.5pt]
			\multicolumn{1}{m{5cm}}{\centering 符号} & \multicolumn{1}{m{5cm}}{\centering 说明} \\
			\midrule[1pt]		
			$P_n$  & 20个站点  \\ 
			$P_n$  & 20个站点  \\ 
		   	$P_n$  & 20个站点  \\ 
			\bottomrule[1.5pt]
		\end{tabular}
		\begin{tablenotes}
		\item 注:表中未说明的符号以首次出现处为准
		\end{tablenotes}
		\end{table}

	\section{问题一模型的建立与求解}
		\subsection{问题描述与分析}
			问题一要求给出总里程最少的管道铺设方案。在村村通自来水工程的连通图$G=(V(G),E(G))$中,每个供水站可以视作一个节点$v \in V$,节点间的供水管道看作边$e \in E$,那么由中心供水站到I级供水站、由I级供水站到II级供水站分别构成了生成树$T_{i}=(V_{i}(T_{i}),E_{i}(T_{i}))$,其中$V_{i}(G)=V_{i}(T_{i})$,$E_{i}(T_{i}) \subset E_{i}(G) , (i=1,2)$,且根据定理(这里引用一下!!!!!),$|E_{i}|=|V_{i}|-1$。由于I级和II级供水站的本质区别是II级供水站不能与中心供水站直接相连,故本问题实质上是一个二阶最小生成树问题。
			我们以两节点间的欧式距离作为边的代价,应用改进的Prim算法(二阶的,想个名字!!!!!!)求解模型。
			其思维流程图如图~\ref{lct}~所示:
			\begin{figure}[H]
				\centering
				\includegraphics[width=\textwidth]{figures/whut.jpg}
				\caption{问题一思维流程图}\label{lct}
			\end{figure}
			
		\subsection{模型的建立}
			同一般的最小生成树问题不同,问题一中的II级供水站生成树优化需待I级供水站的最小连通树生成后才能开始进行。因此,问题一的模型(取个屌点的名字?????)分为两部分,I级供水站和II级供水站先后进行目标优化。
			本问题中,边$e$的代价是两节点$v_i(x_i,y_i), v_j(x_j,y_j)$间的欧式距离,可表示为:
		\begin{gather}
		cost(v_i,v_j)=\sqrt {(x_i-x_j)^2 +(y_i-y_j)^2} ,
		\end{gather}
			在此二阶最小生成树问题中,已知节点和边的关系为:$|E_{i}|=|V_{i}|-1, (i=1,2)$,其中$|E_{i}|$是边的数目,$|V_{i}|$是节点个数。把生成树节点对应的序列作为决策变量,以序列矩阵的形式表示,每一层的最优序列矩阵可以表示为
			\begin{gather*}
			A_k=
			\begin{bmatrix}
			v_{11} &v_{12} \\ 
			v_{21} &v_{22} \\ 
			...&...\\
			v_{|E_{k}|,1}&v_{|E_{k}|,2}\\
			\end{bmatrix}  , k=1,2
			\end{gather*}
			其中,$v_{i1}, v_{i2}(1\leq i \leq|E_{k}|)$分别代表边$e_i$的起始节点和终止节点。
			
			分别将I、II管道总里程作为优化目标,可知总里程是所有边的代价之和,而代价可以表示成节点间的距离,最佳节点序列已经存放在最优序列矩阵中,这些节点距离可以通过节点序列取出。同时,总里程是关于$E_{i}$的函数。因此,在第一层和第二层分别对距离求和,可得目标函数最短路径为:
			\begin{gather}
			F(E)=F_{1}(E)+F_{2}(E),\\
			F_{k}(E)=\sum_{i=1}^{|E_{k}|}cost(v_{i1},v_{i2}),k=1,2
			\end{gather}
			
			此问题的约束条件为问题一中的II级供水站生成树优化不能先于I级供水站的最小连通树生成,对应的数学描述为
			
			以从中心出发铺设的自来水管道总里程最少,结合约束条件,得到自来水管道最短路径:
			\begin{gather}
			min F(E)
			\end{gather}
			\begin{gather*}
			s.t.\left\{\begin{matrix}
			F(E)=F_{1}(E)+F_{2}(E)\\ 
			F_{k}(E)=\sum_{i=1}^{|E_{k}|}cost(v_{i1},v_{i2}),k=1,2\\ 
			cost(v_i,v_j)=\sqrt {(x_i-x_j)^2 +(y_i-y_j)^2}\\ 
			?????
			\end{matrix}\right.
			\end{gather*}
		\subsection{模型的求解}
		为求取管道最小里程和最优路径,我们需要求解每一层的最优序列矩阵。针对问题一,我们采用Prim算法,分别实现由中心供水站到I级供水站、由I级供水站到II级供水站的最小生成树,可以求得每一层的最优序列矩阵$A_k$。
		
		Prim算法的伪代码如下。
		
			\begin{algorithm}[H]
			 	\caption{Procedure of Apriori}  
			 	\LinesNumbered  
			 	\setstretch{0.9}   %设置表的行间距
			 	\KwIn{item data base: $D$\newline
			 		minimum Support threshold: $Sup_{min}$\newline
			 		minimum Confidence threshold: $Conf_{min}$
			 	}
			 	\KwOut{frequent item sets $F$}  
			 	\textbf{Initialize} \newline
			 	iteration $t\leftarrow 1$ \newline
			 	The candidate FIS:$C_{t}=\varnothing$ \newline
			 	The length of FIS:$length=1$ \newline
			 	\For{i=1 to sizeof(D)}
			 	{$I_{i}$=D(i)\newline
			 		n=sizeof($I_{i}$)\newline
			 		\For{j=1 to n}{
			 			\If{$I_{i}(j)\notin C_{t}$ }
			 			{$C_{t}=C_{t}\cup I_{i}(j) $}
			 		}
			 	}
			 	$F_{t}=\left \{ f|f\in C_{t},Sup(f)>Sup_{min}\right \}$\newline
			 	\While{$F\neq \varnothing$}
			 	{ t=t+1\newline 
			 		length=length+1\newline	
			 		$C_{t}\leftarrow $ all candidate of FIS in $F_{t-1}$\newline
			 		$F_{t}=\left \{ f|f\in C_{t},(Sup(f)>Sup_{min})\bigcap (Comf(f)>Conf_{min}) \right\}$\newline
			 	}	
			 	\Return{$F_{t-1}$} 
			\end{algorithm} 
		结果在这里放一些。。。。。。
        \subsection{实验结果及分析}
  1.灵敏度分析;(2.对比分析);3.算法收敛性分析;4.算法时间复杂度分析。。。
	\section{问题二模型的建立与求解}
		\subsection{问题描述与分析}
			问题二要求升级两个II级供水站为I级供水站,使得II级管道里程数最少。
			
			其中,第一层更新后的决策变量仍是最优节点序列,用最优序列矩阵表示为:
			\begin{gather}
			A_1'=
			\begin{bmatrix}
				v_{11} &v_{12} \\ 
				v_{21} &v_{22} \\ 
				...&...\\
				v_{\left | E _{1}\right |,1}&v_{\left | E _{1}\right |,2}\\
				v_{\left | E' _{1}\right |,1}&v_{\left | E'_{1}\right |,2}\\
			\end{bmatrix} ,|E_1'|=|E_1|+1
			\end{gather}
			其中,$A'$是更新后的最优序列矩阵。$|E_1'|$,$|E_1|$分别是更新后和更新前的边集合的模,即边数。升级两个II级供水站为I级后,增加一条边,故满足$|E_1'|=|E_1|+1$。$v_{i1}, v_{i2}(1\leq i \leq|E_{k}|)$分别代表边$e_i$的起始节点和终止节点。
			
			目标函数为II级管道的总里程
			\begin{gather}
			F_{2}(E)=\sum_{i=1}^{|E_{2}|'}cost(v_{i1},v_{i2}),
			\end{gather}

			
    		其思维流程图如图~\ref{lssssct}~所示:

			\begin{figure}[H]
				\centering
				\includegraphics[width=\textwidth]{figures/whut.jpg}
				\caption{问题二思维流程图}\label{lssssct}
			\end{figure}

		\subsection{模型的建立}
		
		\subsection{模型的求解}

        \subsection{实验结果及分析}
        
			结果如下表\ref{zhuanssssasgzai}所示:
			\begin{table}[H]
			\setstretch{1.4}  %设置表的行间距
			\centering		
			\caption{xxxxxxxxxxxxxxxxxxxxx}\label{biao1}
			\begin{tabular}{cc}
			\toprule[2pt]
				\multicolumn{1}{m{5cm}}{\centering xxxxxxx}
				& \multicolumn{1}{m{5cm}}{\centering xxxxxxx}
				\\
				\midrule[1pt]
				xxxxxxx &   909.80\\ 
				xxxxxxx & 	852.60\\ 
			\bottomrule[2pt]	
			\end{tabular}
			\end{table}
  
  			由表\ref{biao1}可知

			其各个小车的运输细节图下图所示:
			\begin{figure}[H]
				\centering
				\subfigure{\includegraphics[height=8cm,width=7.5cm]{figures/whut.jpg}}
				\subfigure{\includegraphics[height=8cm,width=7.5cm]{figures/whut.jpg}}
			\end{figure}	
			\begin{figure}[H]	
				\centering
				\subfigure{\includegraphics[height=8cm,width=7.5cm]{figures/whut.jpg}}
				\subfigure{\includegraphics[height=8cm,width=7.5cm]{figures/whut.jpg}}
				\caption{xxxxxxxxxxxxxxxxxxxxxxxxx}
				\label{fisg}
			\end{figure}

    \section{问题三模型的建立与求解}
    
    
  		\subsection{结果分析}
  
  	\section{灵敏度分析}
 
  	\section{模型的评价}
		\subsection{模型的优点}
			\begin{itemize}                                             
			\item [(1)]
			\item [(2)] 	
			\end{itemize}
		\subsection{模型的缺点}

  		\subsection{模型改进}

  
  
 
	\newpage	%换页符
	%%参考文献
	%\begin{thebibliography}{9}%宽度9
	% \setlength{\itemsep}{-2mm}
	\nocite{*}		%排版未引用的参考文献
	\begin{thebibliography}{9}%宽度9
		\bibitem{1}Jiang B, Zhang L. Research on minimum spanning tree based on prim algorithm[J]. Computer Engineering and Design, 2009, 13.
		\bibitem{2}Graham R L, Hell P. On the history of the minimum spanning tree problem[J]. Annals of the History of Computing, 1985, 7(1): 43-57.
		\bibitem{3}丁建立, 陈增强, 袁著祉. 遗传算法与蚂蚁算法的融合[J]. 计算机研究与发展, 2003, 40(9): 1351-1356.
		\bibitem{4}贺毅朝, 刘坤起, 张翠军, 等. 求解背包问题的贪心遗传算法及其应用[J]. 计算机工程与设计, 2007, 28(11): 2655-2657.
		\bibitem{5}王磊, 潘进, 焦李成. 免疫算法[D]. , 2000.
		\bibitem{6}卢开澄, 卢华明. 图论及其应用[M]. 清华大学出版社有限公司, 2004.
	\end{thebibliography}

	\newpage
	%附录
	\appendix %%附录
	\section{问题一、二代码及其可视化}
		\subsection*{Graph类实现最小生成树算法}
			\begin{lstlisting}[language=python]
			import pandas as pd
			import numpy as np
			import matplotlib.pyplot as plt
			import copy
			import networkx as nx
			from tqdm.notebook import tqdm
			
			class Graph(object):
			def __init__(self, Matrix, add_edge=None, add_node=None):
			self.Matrix = Matrix
			self.nodenum = len(self.Matrix)
			self.edgenum = self.get_edgenum()
			self._weight_ = np.zeros((self.nodenum, self.nodenum))
			self.add_edge = add_edge
			self.add_node = add_node
			
			def get_edgenum(self):
			count = 0
			for i in range(self.nodenum):
			for j in range(i):
			if self.Matrix[i][j] > 0 and self.Matrix[i][j] < 9999:
			count += 1
			return count
			
			def plot_matrix(self, pos=None, figsize=(15,15), title="Pipeline ONE"):
			plt.figure(figsize=(12,9)) 
			self._get_edge()
			G_nx = nx.Graph()
			G_nx2 = nx.Graph()
			if self.add_edge!=None:
			for i in range(self.nodenum):
			for j in range(self.nodenum):
			if self._weight_[i, j]!=0 and i<13 and j<13:
			G_nx.add_edge(i, j)
			if self._weight_[i, j]!=0 and i>0 and j>0:
			G_nx2.add_edge(i, j)
			else:
			for i in range(self.nodenum):
			for j in range(self.nodenum):
			if self._weight_[i, j]!= 0:
			G_nx.add_edge(i, j)
			
			if self.add_edge!=None:
			nx.draw_networkx(G_nx, pos[:len(self.add_edge)+1], alpha=0.85)
			nx.draw_networkx(G_nx2,pos,alpha=0.6,with_labels=False,node_color='slateblue',
			node_shape=".", node_size=100, style='dashed')
			else:
			nx.draw_networkx(G_nx, pos, alpha=0.85)
			GG = nx.Graph()
			GG.add_node(0)
			nx.draw_networkx(GG, {0:pos[0]}, node_color='r',node_shape='*', node_size=1200)
			plt.title(title)
			plt.show() # display
			
			def _get_edge(self):
			edge = self.prim()
			if self.add_node!=None:
			for i in edge:
			for id_j, j in enumerate(self.add_node):
			if i[0]==j:
			i[0] = 13+id_j
			if i[1]==j:
			i[1] = 13+id_j
			for k in edge:
			self._weight_[k[0],k[1]] = self.Matrix[k[0],k[1]]
			if self.add_node!=None:
			for i in edge:
			for id_j, j in enumerate(self.add_node):
			if i[0]==13+id_j:
			i[0] = j
			if i[1]==13+id_j:
			i[1] = j
			return self._weight_
			
			def prim(self, first_node = 0):
			# 存储已选顶点,初始化时可随机选择一个起点
			select = [first_node]
			# 存储未选顶点
			candidate = list(range(0, self.nodenum))
			candidate.remove(first_node)
			if self.add_edge!=None:
			node = []
			select.remove(first_node)
			for i in self.add_edge:
			if i[0] not in node:
			node.append(i[0])
			if i[1] not in node:
			node.append(i[1]) 
			for i in node:
			select.append(i)
			if i in candidate:
			candidate.remove(i)
			# 存储每次搜索到的最小生成树的边
			edge = []+self.add_edge if self.add_edge!=None else []
			
			def min_edge(select, candidate, graph):
			min_weight = np.inf
			v, u = 0, 0
			for i in select:
			for j in candidate:
			if min_weight > graph[i][j]:
			min_weight = graph[i][j]
			v, u = i, j
			return v, u
			
			num = len(self.add_edge)+1 if self.add_edge!=None else 1
			for i in range(num, self.nodenum):
			v, u = min_edge(select, candidate, self.Matrix)
			edge.append([v, u])
			select.append(u)
			candidate.remove(u)
			if self.add_node!=None:
			for i in edge:
			for id_j, j in enumerate(self.add_node):
			if i[0]==13+id_j:
			i[0] = j
			if i[1]==13+id_j:
			i[1] = j
			return edge
			\end{lstlisting}
			
		\subsection*{问题一代码实现及可视化}
			\begin{lstlisting}[language=python]
			def distance(x1,y1,x2,y2):
			return np.sqrt((x1-x2)**2+(y1-y2)**2)
			
			def fix(x):
			if x.startswith('A'):
			return 0
			return 1 if x.startswith('V') else 2
			
			def get_xy(i,j=0, add_node = None):
			pos = []   # 元组中的两个数字是第i(从0开始计数)个点的坐标
			if add_node==None:
			for k in range(j, i):
			pos.append((data['X坐标'].loc[k], data['Y坐标'].loc[k]))
			else:
			for k in range(j, i):
			pos.append((data['X坐标'].loc[k], data['Y坐标'].loc[k]))
			for k in add_node:
			pos.append((data['X坐标'].loc[k], data['Y坐标'].loc[k]))
			return pos
			
			weight_array = np.zeros((181,181))
			data = pd.read_excel('/content/drive/My Drive/competitions/CMCM/demo1/data.xlsx')
			data['类型'] = data['类型'].apply(lambda x:fix(x))
			
			# 初始化权重矩阵
			for i in tqdm(range(181)):
			for j in range(181):
			point_i = data[data['序号']==i]
			point_j = data[data['序号']==j]
			weight_array[i][j] = distance(point_i['X坐标'].values,
			point_i['Y坐标'].values,
			point_j['X坐标'].values,
			point_j['Y坐标'].values)
			if (i==0 and j>12) or (j==0 and i>12):
			weight_array[i][j]=0
			weight_array[weight_array==0] = 10000
			
			weight_array_A = weight_array[:13,:13]
			G_A = Graph(weight_array_A)
			pos_A = get_xy(G_A.nodenum)
			edge_A = G_A.prim(first_node=0)
			G_A.plot_matrix(pos_A)
			
			G = Graph(weight_array, edge_A)
			print('节点数据为%d,边数为%d\n'%(G.nodenum, G.edgenum))
			pos = get_xy(G.nodenum)
			edge = G.prim()
			G.plot_matrix(pos, title="Pipeline TWO")
			
			sum = 0
			for p in edge:
			i,j=p[0],p[1]
			sum = sum+weight_array[i][j]
			sum
			\end{lstlisting}
			
		\subsection*{问题二代码实现及可视化}
			\begin{lstlisting}[language=python]
			_max = (0,0,0)
			_max2 = (0,0,0)
			for ed in edge:
			i,j = ed[0],ed[1]
			if _max[0] < weight_array[i][j] and not (i<13 and j<13):
			_max = (weight_array[i][j], i, j)
			for ed in edge:
			i,j = ed[0],ed[1]
			if i!=126 and j!=125:
			if _max2[0] < weight_array[i][j] and not (i<13 and j<13):
			_max2 = (weight_array[i][j], i, j)
			
			_max, _max2
			
			_weight_ = G._get_edge()
			plt.figure(figsize=(12,9)) 
			G_nx = nx.Graph()
			G_nx2 = nx.Graph()
			G_nx3 = nx.Graph()
			G_nx4 = nx.Graph()
			GG = nx.Graph()
			for i in range(G.nodenum):
			for j in range(G.nodenum):
			if _weight_[i, j]!=0 and i<13 and j<13:
			G_nx.add_edge(i, j)
			if _weight_[i, j]!=0 and i>0 and j>0:
			G_nx2.add_edge(i, j)
			# G_nx3.add_edge(126, 125)
			# G_nx4.add_edge(88, 89)
			G_nx3.add_node(125)
			G_nx4.add_node(89)
			
			nx.draw_networkx(G_nx3, {125:pos[125],},node_color='r',
			node_size=300, node_shape='.',with_labels=False, style='dashed')
			nx.draw_networkx(G_nx4, {89:pos[89],},node_color='r',
			node_size=300, node_shape='.',with_labels=False, style='dashed')
			
			nx.draw_networkx(G_nx, pos[:len(G.add_edge)+1], alpha=0.85)
			nx.draw_networkx(G_nx2,pos,alpha=0.6,with_labels=False,node_color='slateblue',
			node_shape=".", node_size=100, style='dashed')
			GG = nx.Graph()
			GG.add_node(0)
			nx.draw_networkx(GG, {0:pos[0]}, node_color='r',node_shape='*', node_size=1200)
			
			plt.title("Upgrade Two Secondary Pipes")
			plt.show() 
			\end{lstlisting}
			
		\section{问题三代码及其可视化}
			\subsection*{改进prim算法}
			\begin{lstlisting}[language=python]
			def prim_pro(Matrix, add_edge, first_node=0, cost=40):
			nodenum = len(Matrix)
			
			select = [first_node]
			candidate = list(range(0, nodenum))
			candidate.remove(first_node)
			node = []
			for i in add_edge:
			if i[0] not in node:
			node.append(i[0])
			if i[1] not in node:
			node.append(i[1]) 
			node_A = {}
			for i in node:
			node_A[i] = [0,[i]]  # 存储形式是[dis, [v,u1,u2...]]
			for i in node:
			select.append(i)
			if i in candidate:
			candidate.remove(i)
			# 存储每次搜索到的最小生成树的边
			edge = []+add_edge
			loss = []
			def min_edge(select, candidate, graph):
			min_weight = np.inf
			v, u = 0, 0
			for i in select:
			for j in candidate:
			if min_weight > graph[i][j] and (i,j) not in loss:
			min_weight = graph[i][j]
			v, u = i, j
			for x in node:
			if v in node_A[x][1]:
			if node_A[x][0]+min_weight<=cost:
			node_A[x][0] = node_A[x][0]+min_weight
			node_A[x][1].append(u)
			else:
			loss.append((v,u))
			return False, v, u
			return True, v, u
			
			num = len(add_edge)+1
			test = 0
			while test<num*(nodenum-num):
			flag, v, u = min_edge(select, candidate, Matrix)
			if flag:
			edge.append([v, u])
			select.append(u)
			candidate.remove(u)
			else:
			test = test+1
			return edge, candidate, node_A
			\end{lstlisting}
			\subsection*{免疫遗传算法及其可视化}
			\begin{lstlisting}[language=python]
			COST = 40
			e, c, node_A = prim_pro(weight_array, edge_A, cost=COST)
			_size = len(c)
			_size#, c
			
			_weight_ = np.zeros((181,181))
			for k in e:
			_weight_[k[0],k[1]] = weight_array[k[0],k[1]]
			plt.figure(figsize=(12,9)) 
			G_nx = nx.Graph()
			G_nx2 = nx.Graph()
			G_nx3 = nx.Graph()
			
			for i in range(181):
			for j in range(181):
			if _weight_[i, j]!=0 and i<13 and j<13:
			G_nx.add_edge(i, j)
			if _weight_[i, j]!=0 and i>0 and j>0:
			G_nx2.add_edge(i, j)
			nx.draw_networkx(G_nx, pos[:len(edge_A)+1], alpha=0.85)
			nx.draw_networkx(G_nx2,pos,alpha=0.6,with_labels=False,node_color='slateblue',
			node_shape=".", node_size=100, style='dashed')
			GG = nx.Graph()
			GG.add_node(0)
			nx.draw_networkx(GG, {0:pos[0]}, node_color='r',node_shape='*', node_size=1200
			,label='Central pipeline')
			p = {}
			for i in c:
			G_nx3.add_node(i)
			p[i] = pos[i]
			nx.draw_networkx(G_nx3, p, node_color='purple', node_size=120,with_labels=False,
			node_shape=",",label='Isolated point')
			
			plt.show() # display
			
			def fun(node_B, learning = 100):
			node_B_index = []
			for i in range(_size):
			if node_B[i]:
			node_B_index.append(i+13)
			node_A_index = list(range(0, 13))+node_B_index
			weight_array_A = np.zeros((len(node_A_index),len(node_A_index)))
			for id_i,i in enumerate(node_A_index):
			for id_j,j in enumerate(node_A_index):
			weight_array_A[id_i][id_j] = weight_array[i][j]
			G_A = Graph(weight_array_A)
			edge_AA = G_A.prim(first_node=0)
			e, c, node_A = prim_pro(weight_array, edge_AA, cost=COST)
			return np.sum(node_B)+learning*len(c)
			
			def mutate(node_B1, mu = 0.05):
			l = len(node_B1)
			for i in range(l):
			p = np.random.random()
			if p < mu:
			node_B1[i] = 0 if node_B1[i]==1 else 0 
			return node_B1
			
			def cross(node_B1,node_B2):
			le = round(len(node_B1)/3)
			temp = node_B1[:le].copy()
			node_B1[:le] = node_B2[:le]
			node_B2[:le] = temp
			return node_B1,node_B2
			
			def select(geti):
			sss = []
			for i in tqdm(geti):
			sss.append(fun(i))
			idx = [i for i,v in sorted(enumerate(sss), key=lambda x:x[1])]
			geti2 = []
			for i in idx:
			geti2.append(geti[i])
			le = round(len(geti)*3/4)
			# le = len(geti)-1
			g = []
			for i in range(le):
			g.append(geti2[i])
			return g, sss, idx
			
			add_node = [89]
			node_A_index = list(range(0, 13))+add_node
			weight_array_A = np.zeros((len(node_A_index),len(node_A_index)))
			for id_i,i in enumerate(node_A_index):
			for id_j,j in enumerate(node_A_index):
			weight_array_A[id_i][id_j] = weight_array[i][j]
			G_A = Graph(weight_array_A,add_node=add_node)
			edge_AA = G_A.prim(first_node=0)
			
			pos_A = get_xy(13,add_node=add_node)
			G_A.plot_matrix(pos_A)
			e, c, node_A = prim_pro(weight_array, edge_AA, cost=COST)
			
			_weight_ = np.zeros((181,181))
			for k in e:
			_weight_[k[0],k[1]] = weight_array[k[0],k[1]]
			plt.figure(figsize=(12,9)) 
			G_nx = nx.Graph()
			G_nx2 = nx.Graph()
			G_nx3 = nx.Graph()
			pos_A = get_xy(13,add_node=add_node)
			for i in range(181):
			for j in range(181):
			if _weight_[i, j]!=0 and i<13 and j<13:
			G_nx.add_edge(i, j)
			if _weight_[i, j]!=0 and i>0 and j>0:
			G_nx2.add_edge(i, j)
			nx.draw_networkx(G_nx, pos_A, alpha=0.85)
			nx.draw_networkx(G_nx2,pos,alpha=0.6,with_labels=False,node_color='slateblue',
			node_shape=".", node_size=100, style='dashed')
			GG = nx.Graph()
			GG.add_node(0)
			nx.draw_networkx(GG, {0:pos[0]}, node_color='r',node_shape='*', node_size=1200
			,label='Central pipeline')
			p = {}
			for i in add_node:
			G_nx3.add_node(i)
			p[i] = pos[i]
			nx.draw_networkx(G_nx3, p, node_color='gold', node_size=150,with_labels=False,
			node_shape=",",label='Isolated point')
			
			plt.show() # display
			\end{lstlisting}
%			\lstinputlisting[language={python},numbers=left,numberstyle=\tiny,
%			rulesepcolor=\color{red!20!green!20!blue!20},  
%			keywordstyle=\color{blue!70!black},  
%			commentstyle=\color{blue!90!},  
%			basicstyle=\ttfamily] {./code/demo.py}

\end{document}
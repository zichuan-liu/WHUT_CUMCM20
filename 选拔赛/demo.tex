\documentclass{whutmod}
\usepackage{metalogo}
\usepackage{float}
\usepackage{subfigure} 
\usepackage{url}
\usepackage{booktabs}
\bibliographystyle{unsrt}
\team{23}
\membera{刘子川}
\joba{编程}
\memberb{程宇}
\jobb{建模}
\memberc{祁成}
\jobc{建模}
\hypersetup{
	colorlinks=true,
	linkcolor=black
}

\title{基于xx模型}
\tihao{1} 

\begin{document}

%\maketitle

	%目录
	\thispagestyle{empty}
	\tableofcontents
	\setcounter{page}{0}                                               
	\newpage	%换页符
	
	\section{问题重述}	
	\subsection{问题背景}
   在物资调运过程中,完成指定点的调运任务是最基本
   的要求,在完成基本的任务之外,往往有更高的追求,比如
   如何使总运费最省?怎样才能使得运输时间最短?如何选
   择运输路径使得运输总距离最短等等。这些更高的追求往
   往是企业期望达到的目标,为了解决这些类似问题,有必
   要对物资调运的过程进行数学模型的建立,以期通过模型
   来理解和分析物资调运的过程,并为其找到解决的方法。
   现以具体的食品调运案例进行分析研究。
    
    某食品公司有19个食品销售点,销售点的地理坐标和每天的需求量见附件。每天凌晨都要从仓库(第20号站点)出发将食品运至每个销售点,运送物品后最终返回仓库。现有运送食品的运输车,每台车每日工作 4小时,运输车重载运费2元/吨公里,并且假定街道方向均平行于坐标轴,任意两站点间都可以通过一次拐弯到达。

	\subsection{问题概述}
    围绕相关附件和条件要求,研究食品运输车在各仓库间的调度方案,依次提出以下问题:
		 
	
	\textbf{问题一:}若只有一辆载重100吨的大型运输车,运输车平均速度为40公里/小时,每个销售点需要用20分钟的时间下货,空载费用0.6元/公里。它送完所有食品并回到仓库,求最少需要时间及其对应的总距离,总运费。
	
	\textbf{问题二:}有一种小型运输车,运输车平均速度为50公里/小时,每个销售点需要用5分钟的时间下货,载重为6吨,空载费用0.4元/公里;要使它们送完所有食品并回到仓库,运输车应如何调度使总体调度效率最高? 
	
	\textbf{问题三:}如果有载重量为4吨、6吨两种运输车,空载费用分别为0.2、0.4元/公里,其他条件均相同,又如何安排车辆数和调度方案。

	
	\section{模型假设}
	\begin{itemize}                                             
		\item [(1)] 为保证预测结果精确性,假设题目所给出数据真实可信。
		\item [(2)] 假设重点防控的区域和人群中,发病、死亡人数的增长率比其基数更加重要
	\end{itemize}
		
	\section{符号说明}
	\begin{table}[H]
	\label{biao} \centering
	\begin{tabular}{cc}
		\toprule[1.5pt]
		\multicolumn{1}{m{5cm}}{\centering 符号} & \multicolumn{1}{m{5cm}}{\centering 说明} \\
		\midrule[0.5pt]		
		$X^{(i)}$  & 人数时间序列  \\ 
		$a$  &  发展灰度 \\ 
		$u$  &  内生控制灰度\\
		\bottomrule[1.5pt]
	\end{tabular}
\end{table}

	\section{问题一模型的建立与求解}
    \subsection{问题描述与分析}

    其思维流程图如图~\ref{lct}~所示:

       \begin{figure}[H]
	   	\centering
	   	\includegraphics[width=\textwidth]{figures/sanrenf.png}
	   	\caption{问题一思维流程图}\label{lct}
	   \end{figure}

   
	    \subsection{模型的建立}
	    \subsubsection{灰度预测GM(1,1)}
	    设2004-2016年总发病人数为时间序列:
	     \begin{gather*}
	    X^{(0)}=[x^{(0)}(1),x^{(0)}(2),\cdots,x^{(0)}(13)]
	    \end{gather*}
	    
	    

	  其误差状态区间如表~\ref{ff}~所示:
	  	 \begin{table}[H]
	  	\centering\caption{发病人数状态区间划分}\label{ff}
	  	\begin{tabular}{cccc}
	  		\toprule[1.5pt]
	  		\multicolumn{1}{m{2cm}}{\centering 状态}
	  		& \multicolumn{1}{m{3cm}}{\centering $E_{1}$}
	  		& \multicolumn{1}{m{3cm}}{\centering $E_{2}$}
	  		& \multicolumn{1}{m{3cm}}{\centering $E_{3}$}
	  		\\
	  		\midrule[0.5pt]
	  		残差区间 &  $[-66389,-22130]$  &$(-22130,22130]$ & $(22130,66389]$   \\ 
	  		\bottomrule[1.5pt]	
	  	\end{tabular}
	  \end{table}  

	  
	  \section{问题二模型的建立与求解}
	  \subsection{问题描述与分析}

	
	  \subsection{模型的建立}
	  
	     
    \subsection{模型的求解}

 
  
  
  
  \section{灵敏度分析}
  
  
  
  \section{模型的评价}
  \subsection{模型的优点}
  \begin{itemize}                                             
  	\item [(1)] 利用马尔可夫模型改进后的灰度预测值与实际值拟合度更高,波动性保持一致,预测的效果更好。
  	\item [(2)] 针对支持向量回归参数选取,利用灰色关联度筛选合适指标,相较于主观选取指标具有客观性、严谨性。	
  \end{itemize}
  \subsection{模型的缺点}
  
  问题一、二中的灰色预测模型只能做短期预测,并不适用于长期预测。
  \subsection{模型改进}
  
  可以通过序列最小优化算法(Sequential Minimal Optimization,SMO)作为样本的训练算法,进而建立序列最小优化支持向量回归模型,从而减小算法复杂度,提高算法的求解速度。
  
  
  
 
	\newpage	%换页符
	%%参考文献
	%\begin{thebibliography}{9}%宽度9
	% \setlength{\itemsep}{-2mm}
	\nocite{*}		%排版未引用的参考文献
%\bibliography{wenxian.bib}
%	%参考文献添加到wenxian.bib里,再引用
%	
\begin{thebibliography}{9}%宽度9
	\bibitem{bib:one}张斯嘉, 郭建胜, 钟夫, 等. 基于蝙蝠算法的多目标战备物资调运决策优化[J]. 火力与指挥控制, 2016, 41(1): 58-61.
	\bibitem{bib:2}李健, 张文文, 白晓昀, 等. 基于系统动力学的应急物资调运速度影响因素研究[J]. 系统工程理论与实践, 2015, 35(3): 661-670.	
	\bibitem{3}Wang J, Ersoy O K, He M, et al. Multi-offspring genetic algorithm and its application to the traveling salesman problem[J]. Applied Soft Computing, 2016, 43: 415-423.
	\bibitem{4}陶丽华, 马振楠, 史朋涛, 等. 基于 TSP 问题的动态蚁群遗传算法[J]. 机械设计与制造, 2019 (12): 39.
\end{thebibliography}

	\newpage
	%附录
	\appendix %%附录

\section{模型的代码实现}

\subsection{GATSP--matlab源代码}
\begin{lstlisting}[language=matlab]
clear;
w=20;g=100;d=19;%w为种群数,g代数,d维数
G(1:w,1:d)=0;%初始化空间
for i=1:w%初始化
c=randperm(d);
for t=1:20
flag=0;
for t1=1:d-1
for t2=t1+1:d
cl=c;
cl(t1:t2)=cl(t2:-1:t1);
if distan(cl)<distan(c)
c=cl;
flag=1;
end
end
end
if flag==0
G(i,1:d)=c;break
end
end
end
for k=1:g %进入遗传循环
A=G;%预备交叉阵
c=randperm(w);%配对序列
%c=1:w;
for i=1:2:w %交叉
F1=ceil(rand*d);%交叉点1
F2=ceil(rand*d);%交叉点2
while(F1==F2)
F2=ceil(rand*d);
end
if(F1>F2)%交叉地址调序
tem=F1;
F1=F2;
F2=tem;
end
j=0;t=1;%计数标值
while(j~=d+F1-F2-1)%如果剩余基因没完全插入就继续
if(isempty(find(A(c(i),F1:F2)==G(c(i+1),t),1))) %目标基因于交换片段中都不同
j=j+1;
if j<F1 %前半段基因交换
A(c(i),j)=G(c(i+1),t);
A(c(i+1),t)= G(c(i),j);
else %后半段基因交换
A(c(i),j+F2-F1+1)=G(c(i+1),t);
A(c(i+1),t)=G(c(i),j+F2-F1+1);
end
end
t=t+1;
end
end
by=[];
while isempty(by)
by=find(rand(1,w)<0.3);%变异地址
end
B=G(by,1:d);%预备变异阵
for j=1:length(by)
bw=sort(ceil(rand(1,2)*d));%变异基因节点
B(j,bw(1))=G(j,bw(2));%单点基因交换
B(j,bw(2))=G(j,bw(1));
end
GG=[G;A;B];%GG为选择阵    
clear A; clear B;%清除数据防止规格保存
m=size(G,1);%选择阵个体数
long(1:m)=0;%目标函数初始化
for i=1:m%计算函数
long(i)=distan(GG(i,:));
end
[slong,ind]=sort(long(1:m));%目标函数排序
for i=1:w%精英选择
G(i,:)=GG(ind(i),:);
end
clear GG;%清除数据防止规格保存
end      
\end{lstlisting}


\subsection{MGATSP--matlab源代码}
\begin{lstlisting}[language=matlab]
clear;
for pp=6:13 %6:13
for ppp=1:100
n=pp;w=20;g=100;d=19+n-1;%n为车数,w为种群数,g代数,d维数
G(1:w,1:d)=0;%初始化空间
for i=1:w%初始化
c=randperm(d);
for t=1:20
flag=0;
for t1=1:d-1
for t2=t1+1:d
cl=c;
cl(t1:t2)=cl(t2:-1:t1);
if price(cl)<price(c)
c=cl;
flag=1;
end
end
end
if flag==0
G(i,1:d)=c;break
end
end
end
for k=1:g %进入遗传循环
A=G;%预备交叉阵
c=randperm(w);%配对序列
%c=1:w;
for i=1:2:w %交叉
F1=ceil(rand*d);%交叉点1
F2=ceil(rand*d);%交叉点2
while(F1==F2)
F2=ceil(rand*d);
end
if(F1>F2)%交叉地址调序
tem=F1;
F1=F2;
F2=tem;
end
j=0;t=1;%计数标值
while(j~=d+F1-F2-1)%如果剩余基因没完全插入就继续
if(isempty(find(A(c(i),F1:F2)==G(c(i+1),t),1))) %目标基因于交换片段中都不同
j=j+1;
if j<F1 %前半段基因交换
A(c(i),j)=G(c(i+1),t);
A(c(i+1),t)= G(c(i),j);
else %后半段基因交换
A(c(i),j+F2-F1+1)=G(c(i+1),t);
A(c(i+1),t)=G(c(i),j+F2-F1+1);
end
end
t=t+1;
end
end
by=[];
while isempty(by)
by=find(rand(1,w)<0.3);%变异地址
end
B=G(by,1:d);%预备变异阵
for j=1:length(by)
bw=sort(ceil(rand(1,2)*d));%变异基因节点
B(j,bw(1))=G(j,bw(2));%单点基因交换
B(j,bw(2))=G(j,bw(1));
end
GG=[G;A;B];%GG为选择阵    
clear A; clear B;%清除数据防止规格保存
m=size(G,1);%选择阵个体数
long(1:m)=0;%目标函数初始化
for i=1:m%计算函数
long(i)=price(GG(i,:));
end
[slong,ind]=sort(long(1:m));%目标函数排序
for i=1:w%精英选择
G(i,:)=GG(ind(i),:);
end
clear GG;%清除数据防止规格保存
end 
result(pp-5,ppp)=long(1);
XXX(pp-5,ppp,1:d)=G(1,1:d);
end
end
\end{lstlisting}
\subsection{distan--matlab源代码}
\begin{lstlisting}[language=matlab]
function f=distan(X)
n=size(X,2);
a=[3	2
1	5
5	4
4	7
0	8
3	11
7	9
9	6
10	2
14	0
2	16
6	18
11	17
15	12
19	9
22	5
21	0
27	9
15	19];
f=sum(abs(a(X(1),:)-10));%距离值初始化
for i=1:n-1%计算距离和
f=f+sum(abs(a(X(i+1),:)-a(X(i),:)));
end
f=f+sum(abs(a(X(n),:)-10));%头尾固定
\end{lstlisting}

\section{数据可视化的实现}
\subsection{第一问画图--python源代码}
\begin{lstlisting}[language=python]
from pylab import *
mpl.rcParams['font.sans-serif'] = ['SimHei']

dict = {"1":[3,2], "2":[1,5], "3":[5,4],"4":[4,7], "5":[0,8],"6":[3,11],"7":[7,9],
"8":[9,6],"9":[10,2], "10":[14,0],"11":[2,16], "12":[6,18],"13":[11,17],"14":[15,12],
"15":[19,9],"16":[22,5], "17":[21,0],"18":[27,9], "19":[15,19],"20":[10,10],}
x_axis_data = []
y_axis_data = []
road = [20,8,3,4,5,2,1,9,10,17,16,18,15,14,19,13,12,11,6,7,20]
x_tem = []
y_tem = []
for i in range(len(road)):
x = str(road[i])
print(dict[x])
x_axis_data.append(dict[x][0])
y_axis_data.append(dict[x][1])

try:
x_tem.append(dict[str(road[i+1])][0])
y_tem.append(dict[str(road[i])][1])
except:
pass

x_ = []
y_ = []

for i in range(len(x_tem)):
x_.append(x_axis_data[i])
y_.append(y_axis_data[i])
x_.append(x_tem[i])
y_.append(y_tem[i])

x_.append(x_axis_data[i+1])
y_.append(y_axis_data[i+1])

plt.plot(x_, y_, 'ro-', color='#4169E1', alpha=0.8, label='路径')

for x, y in zip(x_axis_data, y_axis_data):
plt.text(x, y+0.3, '({},{})'.format(x,y), ha='center', va='bottom', fontsize=10.5)

# plt.legend(loc="road")
plt.xlabel('X轴/km')
plt.ylabel('Y轴/km')

# plt.show()
plt.savefig('demo.jpg')  # 保存该图片
\end{lstlisting}
\subsection{第二问画图--python源代码}
\begin{lstlisting}[language=python]
from pylab import *
mpl.rcParams['font.sans-serif'] = ['SimHei']
# import matplotlib.pyplot as plt
import numpy
import matplotlib.colors as colors
import matplotlib.cm as cmx

dicts = {"1":[3,2], "2":[1,5], "3":[5,4],"4":[4,7], "5":[0,8],"6":[3,11],"7":[7,9],
"8":[9,6],"9":[10,2], "10":[14,0],"11":[2,16], "12":[6,18],"13":[11,17],"14":[15,12],
"15":[19,9],"16":[22,5], "17":[21,0],"18":[27,9], "19":[15,19],"0":[10,10],}
x_axis_data = []
y_axis_data = []

cars = [14,0,18,15,0,10,9,0,17,16,0,12,11,0,4,2,3,8,0,5,1,0,19,0,6,7,0,13]

c_ = []
x = [0]
for j in range(len(cars)):
x.append(cars[j])
if cars[j]==0:
c_.append(x)
x = [0]

print(c_)
cmap = plt.cm.jet
cNorm = colors.Normalize(vmin=0, vmax=len(c_))
scalarMap = cmx.ScalarMappable(norm=cNorm, cmap=cmap)
for fff in range(len(c_)):
###########
x_axis_data = []
y_axis_data = []
road = c_[fff]
x_tem = []
y_tem = []
for i in range(len(road)):
x = str(road[i])
x_axis_data.append(dicts[x][0])
y_axis_data.append(dicts[x][1])

try:
x_tem.append(dicts[str(road[i + 1])][0])
y_tem.append(dicts[str(road[i])][1])
except:
pass

x_ = []
y_ = []

for i in range(len(x_tem)):
x_.append(x_axis_data[i])
y_.append(y_axis_data[i])
x_.append(x_tem[i])
y_.append(y_tem[i])


colorVal = scalarMap.to_rgba(fff)
x_.append(x_axis_data[i + 1])
y_.append(y_axis_data[i + 1])
plt.plot(x_, y_, 'o-', alpha=0.8)
for i in range(0,len(x_)-1):
plt.arrow(x_[i], y_[i], x_[i+1] - x_[i], y_[i+1] - y_[i],
length_includes_head=True, head_width=0.3, lw=2,
color=colorVal)

for x, y in zip(x_axis_data, y_axis_data):
plt.text(x, y + 0.3, '({},{})'.format(x, y),)

plt.xlabel('X轴/km')
plt.ylabel('Y轴/km')

# plt.show()
plt.savefig('demo.jpg')  # 保存该图片
\end{lstlisting}
\end{document}